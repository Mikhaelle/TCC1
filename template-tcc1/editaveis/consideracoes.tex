\chapter[Suporte Tecnológico]{Suporte Tecnológico}

\section{Desenvolvimento da Aplicação}

\subsection{Flutter}

O Fluter é um kit de ferramenas de interface de usuário(UI) grátis e open-souce criado pela Google e lançado em 2017. Ele auxilia na criação de aplicaivos nativos para dispositivos mobile, web e desktop a partir de uma única base de código. Isso significa que é possível criar uma aplicação para diferentes sistemas operacionais(IOS e Android) utilizando um unico código \cite{flutter2017}.

Ele é composto por um SDK(Software Development Kit) e um framework. O SDK é uma coleção de ferramentas que ajudam o desenvolvedor a desenvolver a aplicação e executá-la em plataformas específicas. Essas ferramentas incluem bibliotecas, documentação, exemplos de códigos, processos, guias, compiladores, entre outras coisas. Já o framework é uma coleção de elementos da UI que são reutilizaveis e podem ser personalizados para as necessidades expecíficas da aplicação\cite{flutter2017}.

\subsubsection{Dart}

Dart é a linguagem de programação utilizada no flutter e também foi criada pela Google em 2011. É uma linguagem focada para desenvolvimento front-end e do tipo orientada a objeto\cite{flutter2017}.

\subsubsection{Firebase}

Firebase é uma plataforma desenvolvida pelo Google para a criação de aplicativos web e móveis. Era originalmente uma empresa independente fundada em 2011. Em 2014, o Google adquiriu a plataforma e agora é sua ferramenta principal para o desenvolvimento de aplicativos. O Firebase contêm funcionalidade como, análises, bancos de dados, mensagens e relatórios de erros, garantindo mais agilidade no desenvolvimento de aplicativos\cite{firebase2011}.

\section{Engenharia de Software}

\subsection{Gerenciamento do Projeto}

\subsubsection{Trello}


O Trello é uma aplicação web baseado no sistema Kanbam que auxilia no gerenciamento de tarefas para times grandes ou pessoas individuais. Originalmente criado pela Fog Creek Software em 2011 e vendida a Atlassian em 2017 \cite{trello2011}. O Trello foi utilizado para organizar as tarefas relacionadas a escrita do TCC e também para manter o registro dos artigos aqui utilizados.

\subsubsection{ZenHub}
O Zenhub é uma ferramenta semelhante ao Trello. É exclusiva para uso junto ao Github e é um poderosa ferramenta para rastreamento, planejamento e relatórios das features de projetos no GitHub. Ele é baseado nas metodologias ageis, como scrum, e é utilizado em projetos ágeis. Com ele é possível planejar roteiros, usar quadros de tarefas e gerar relatórios automatizados diretamente do repositório do GitHub \cite{zenhub2020}. Nesse projeto será utilizado para organizar as tarefas relacionadas ao desenvolvimento do aplicativo Mina.


\subsubsection{Slack}

\subsection{Gerenciamento de Desenvolvimento}

\subsubsection{Visual Studio Code}

Visual Studio Code (VSCODE, 2017) é um editor de texto ou código fonte da Microsoft(MICROSOFT, 2017) que possui ferramentas poderosas que auxiliam no desenvolvimento de software, como por exemplo conclusão e depuração de código IntelliSense(Sugestões inteligentes de auto preenchimento de algum parâmetro ou atributo no código). Além disso, o editor suporta diversas linguagens de programação, e possui um acervo grande de plugins, visando tornar seu trabalho mais eficiente

\subsubsection{Linux Mint}
(UBUNTU, 2017) é uma distribuição Linux totalmente gratuita e de código aberto. Este sistema operacional, patrocinado pela Canonical(CANONICAL, 2017), foi desenvolvido utilizando o kernel linux em seu núcleo.

\subsection{Gerenciamento de Configuração}

\subsubsection{Git}
Git é uma ferramenta para controle de versão distribuída sob a licença GNU General Public License version 2.0, uma licença open source. Traz benefícios como velocidade, garantia da integridade dos dados e suporte para fluxos de trabalho distribuídos e não-lineares (CHACON; STRAUB, 2014, pág. 31). Por estas razões, o Git versão 2.7.4 foi utilizado para versionamento do código fonte e para a parte escrita do TCC.

\subsubsection{GitHub}

O GitHub é uma plataforma de desenvolvimento de software. Nesta plataforma, é possível hospedar e analisar códigos, gerenciar projetos, e construir software colaborando com outros desenvolvedores. O GitHub apresenta funcionalidades que apoiam: revisão de código; gerenciamento de projetos; integrações de ferramentas; gerenciamento de equipe; codificação social; documentação, e hospedagem de código. Por se tratar de uma ferra-menta online, a versão do Github utilizada foi a disponível durante o desenvolvimento dotrabalho.

\section{Escrita e Condução da Pesquisa}

\subsection{LaTex}
LaTeX 8 é um sistema de composição de documentos de alta qualidade, que inclui recursos projetados para a produção de documentação técnica e científica. LaTeX é o padrão para a comunicação e publicação de documentos científicos, sendo disponibilizada como software livre. As principais funcionalidades da ferramenta utilizadas foram:
(i) Composição de artigos de periódicos, relatórios técnicos, livros e apresentações de slides;
(ii) Controle sobre documentos contendo seções, referências cruzadas, tabelas e figuras;
(iii) Tipografia de fórmulas matemáticas complexas; (iv) Geração automática de biblio-grafias e índices; e (v) Formatação multilíngue.

Neste trabalho, o LaTeX serviu de apoio à escrita, fazendo uso de todas as suas funcionalidades disponíveis nesse sentido. A ferramenta foi utilizada por intermédio doOverleaf.
\section{Resumo do Capítulo}


\chapter[Introdução]{Introdução}

\section{Contextualização}

O ciclo menstrual feminino começou a ser pesquisado, cientificamente, na década de 1930 \cite{frank1931} e em 1973 Barbara Sommer revisou toda a lieratura existente e concluiu, naquela época, que não existia evidencias que a fase do ciclo menstrual influenciava em mudanças na cognição e performance motora por causa problemas na metodolodia. Com o avanço da ciência e da medicina, métodos cada vez mais sofisicados e acessíveis para análise hormonal possibilitaram o avanço dos estudos nessa área que ainda intriga muios cientistas da medicina, psicologia e a sociedade em geral.

Em 2005, estratégias e métodos foram estabelecidos para estudar o ciclo menstrual e obter a classificação correta das fases do ciclo\cite{becker2005}. Essas estratégias e métodos utilizam de medidas hormonais, temperatura corporal basal(TCB) e avaliações baseadas em calendário. Como a concentração de estradiol e progesterona variam muito de mulher para mulher, utilizá-lo como medida única pode ser insuficiente para determinar individualmente a fase do ciclo \cite{poroma2014}.

Tendo como teoria que os hormônios podem influênciar a vida das mulheres, muitos estudos têm sido realizados tentando determinar a influência da fase do ciclo menstrual na capacidade cognitiva, motora e emocional das mulheres. Em 2014, \citeonline{poroma2014} realizaram um levantamento da literatura existente que relacionam o ciclo reprodutivo feminino com as áreas de tarefas cognitivas como, habilidades espacial, visual, verbal, controle cognitivo, e aspectos emocionais.

O ciclo menstrual idealizado tem 28 dias, mas pode variar entre 21 e 35 dias \cite{lenton1984a} e começa a ser contabilizado a partir do primeiro dia da menstruação. O ciclo é dividido principalmente em 2 fases, a fase folicular e a fase lútea \cite{brondin2008}.

A fase folicular é contabilizada a partir do promeiro dia da menstruação até o dia de pico do hormônio luteinizante(LH). Há uma diminuição significativa do comprimento da fase folicular de acordo com a idade das mulheres. Normalmente mulheres de 18 a 24 anos tem a fase com o comprimento de 14 dias e mulheres de 40 a 44 anos tem de 10 dias \cite{lenton1984a}. Em mulheres jovens a diferença no tamanho do ciclo é normalmente provocada por ciclo mais curtos ou mais longos na fase folicular \cite{lenton1984a}.

A fase folicular é caracterizada pelo desenvolvimento folicular, em resposta ao aumento do hormônio folículo-estimulante(FSH) que começa a ser produzido no início da fase. Quando um folículo é selecionado o FSH diminui gradativamente e progressivamente a produção de estradiol começa a aumentar. Quando o estradiol chega ao pico, 12 a 24 horas depois o LH surge e a ovulação ocorre tipicamente de 10 a 12h depois do surgimento do LH \cite{fritz2010}. Clinicamente é possível determinar o ciclo ovulatório pelo surgimento do LH e a secreção de progesterona da fase lútea \cite{fritz2010}, também é possível determiná-lo pelo aumento da temperatura basalcorporal. Na ovulação, o folículo se transforma num corpus luteum que é capaz de sintetizar estradiol e progesterona e está pronto para ser fecundado.

Agora na fase lútea, a progesterona no estágio da ovulação é necessária para preparar o endométrio para a chegada do óvulo no caso de concepção, e o pico da progesterona se dá normalmente por volta do vigésimo primeiro dia do ciclo \cite{nikas2003}. Caso não haja fecundação, a progesterona decai progressivamente e causa novamente a menstruação, continuando assim o ciclo.

O estradiol e a progesterona são altamente lipofílicos, ou seja, se dissolvem em gordura, óleos e lipídios em geral, e facilmente atravessam a barreira sangue-cérebro. Estudos em animais e estudos post-mortem em mulheres na idade reprodutiva e na menopausa indicaram que esses hormônios estavam acumulados no cérebro \cite{bixo1997}. Os receptores desses hormônios estão presentes em áreas cerebrais associadas à reprodução, função cognitiva e processamento emocional, como o hipotálamo e o sistema límbico \cite{gruber2002, brinton2008}.

 No estudo de \citeonline{poroma2014}, na parte de habilidade espacial visual, dos 23 autores estudados, cinco relataram uma melhora nas habilidades no início da fase folicular e três relataram uma melhora nas habilidades quando o estradiol estava baixo(fase folicular), o restante não relatou nenhum efeito da fase. Em tarefas verbais, dos 13 autores estudados, dois relataram uma melhora no meio da fase lútea, 1 relatou melhora em mulheres que utilizam anticoncepcional(AC) e dois relataram melhora no final da fase folicular e no final da fase lútea. No aspecto emocional, varios estudos relacionaram as fases com a habilidade de reconhecer emoçoes faciais. Um indicou uma melhor precisão no inicio e no fim da fase folicular \cite{dernl2013}, já vários outros indicaram uma piora na fase lútea, principalmente em reconhecer emoções negativas \cite{gasbarri2008}. Outro estudo também relacionou o aumento de hormônios do stress com a fase lútea \cite{kirschbaum1999}.

Já nos aspectos emocionais e comportamental, no estudo de Rosa e Mainardes levantou-se que durante o período da semana que antece a menstruação e no período da menstruação, as mulheres entrevistadas relataram sentir uma maior alteração psicológica ou comportamental. Essas alterações variam entre variações de humor, irritabilidade, ansiedade entre outros.

Se existe essa influência das fases do ciclo menstrual na vida das mulheres, como então adiquirir o conhecimento do que é influênciado? porque adiquirir esse conhecimento seria importante? e como utilizar essa influência para benefício próprio?

Para a mulher, adiquirir o conhecimento de que fase se está, como, em que e porque seu ciclo influência sua vida, aumentaria significativamente a sua inteligência emocional, já que ela é medida utilizando a \textit{Multifactor Emotional Intelligence Scale} (MEIS) que é composta por 3 habilidades emocionais distintas: perceber, entender e regular emoções. Em palavras gerais a inteligência emocional é \lq \lq A habilidade de monitorar os próprios sentimentos e emoçoes e de outros indivíduos, discriminar entre eles e usar essas informações para guiar o pensamento e ações \rq \rq \cite{salovey1990} e no estudo realizado por \citeonline{lam2002} ele relaciona o impacto da inteligência emocional com a performance individual e neste estudo o autor confirma a hipotese de que a inteligência emocional influencia no desempenho cognitivo individual, e que das três medidas, a percepção e regulação são as que mais influênciam para o aumento da inteligência emocional.
  

Tendo o conhecimento que vários autores afirmam existir uma influência do ciclo menstrual principalmente nas emoções e comportamentos, que existem levantamentos com grupos de mulheres que indicam tais relações e que entender e perceber impactam a inteligência emocional e a performance individual das pessoas, é possível então que exista alguma influência da fase do ciclo menstrual na performace individual das mulheres em algumas tarefas cotidianas? e se sim, como utilizá-la da melhor forma possível? 

Esse estudo busca responder essas perguntas e propõe a realização de um aplicativo com  sistema de recomendação de caráter informativo, que dará, de acordo com o perfil e o ciclo menstrual individual de cada mulher, indicações de tarefas que seriam mais difíceis ou mais fáceis de serem realizadas no período do ciclo que a mulher se encontra e o porque disto.
 
A problemática será levantar os divesos tipos de perfis das mulheres de acordo com a fase do ciclo menstrual delas, descobrir se existe diferença entre mulheres que utilizam ou não métodos hormonais ou tem algum disturbio hormonal e como essas nuances influenciam as tarefas cotidianas delas. Além de conseguir determinar corretamente em qual fase do ciclo a mulher está apenas com o uso do método do calendário. Para isso, será utilizada a literatura existente sobre o assunto, pesquisas préviamente públicadas e pesquisas próprias, em formato de questionário, aplicadas a mulheres em idade reprodutiva.

Tendo a consciência da delicadesa do assunto estudado, vale ressaltar que este estudo não tem como objetivo entrar na discução de diferença entre sexo, não conta com o acompanhamento de um profissional da saúde e é experimental, tendo como objetivo desenvolver um sistema que seja apenas um guia que facilite a vida das mulheres. 

\section{Justificativa}

Para uma mulher pode ser difícil monitorar, identificar padrões e antecipar mudanças físicas, emocionais e comportamentais que existem no decorrer do ciclo e como isso influencia a vida delas. Os aplicativos vieram ajudá-las nessas questões.

Atualmente, existem muitos aplicativos no mercado que estão voltados para a questão reprodutiva e alguns até possuem \textit{features} para adicionar cotidianamente os sintomas sentidos, humor, temperatura corporal, intensidade do fluxo, tipo de muco, se toma medicamento, quando há relações sexuais, entre outras notas. Entretanto,existe uma carência de um sistema que auxilie com informações e recomendações de tarefas baseadas no perfil e fase do ciclo da mulher.

Outra problemática é que apesar do avanço, relativamente poucas descobertas que relacionam a influência hormonal, do ciclo menstrual, na emoção, comportamento e cognição, surgiram como conclusivas \cite{poroma2014}, e que relatam o impacto dessa influência na vida pessoal das mulheres. Portanto esse estudo também trará um levantamento e pesquisa própria sobre as influências do ciclo menstrual relatadas por um grupo de mulheres em idade reprodutiva.

Para ajudar as mulheres a notar padrões, atencipar mudanças, maximixar o auto-conhecimento, a inteligência emocional e a produtividade pesssoal, esse estudo propõe a realização de um aplicativo com um sistema de recomendação com caráter informativo, que dará, de acordo com o perfil e o ciclo menstrual individual de cada mulher, indicações de tarefas que seriam mais difíceis ou mais fáceis de serem realizadas no período do ciclo que a mulher se encontra e o porque disto.



\section{Objetivos}

\subsection{Objetivos Gerais}

O objetivo geral desse trabalho é desenvolver um aplicativo com um sistema de recomendação de tarefas cotidianas baseado no perfil e ciclo menstrual individual de cada mulher, para auxiliar no auto conhecimento e no aumento da inteligência emocional individual.

\subsection{Objetivos Específicos}

Para atingir o objetivo geral, alguns objetivos específicos foram estabelecidos:

\begin{itemize}

        \item Definir um processo de coleta de perfil das mulheres de acordo com seus ciclos menstruais.

        \item Definir um processo de contagem de ciclo para determinar em que fase do ciclo a mulher se encontra. 

        \item Definir um processo de recomendação de tarefas que podem ser  mais facilmente realizadas baseado na fase do ciclo. 

        \item Definir um processo de recomendação de tarefas que podem ser mais dificilmente realizadas baseado na fase do ciclo.

        \item Especificar um sistema de recomendação de atividades baseado no ciclo menstrual individual dos perfis. 

\end{itemize}

\section{Organização dos capítulos}

Este trabalho de conclusão de curso está organizado nos seguintes capítulos:

\begin{itemize}

        \item  Capítulo 1 - Introdução: apresentou Contextualização, justificativa, problema de pesquisa e objetivos;

        \item Capítulo 2 - Referencial teórico: descreve os conceitos que fundamentam o trabalho e o conhecimento necessário para que se compreenda a pesquisa realizada.
        \item Capítulo 3 - Materiais e Métodos: apresenta o plano metodológico adotado e caracteriza o objeto de estudo;

        \item Capítulo 4 - Proposta: Apresenta a proposta deste trabalho.

        \item Capítulo 5 - Resultados finais: apresenta os resultados alcançados.

        \item Capítulo 6 - Conclusão: apresenta a conclusão a cerca dos resultados alcançados, bem como os possíveis trabalhos futuros.
          
\end{itemize}

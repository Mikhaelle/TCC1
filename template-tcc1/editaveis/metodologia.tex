\chapter[Metodologia]{Metodologia}

\section{Metodologia}

\subsection{Classificação da Pesquisa}

\subsubsection{Quanto à abordagem}

\subsubsection{Quanto à Natureza}

\subsubsection{Quanto aos Objetivos}

\subsubsection{Quanto aos procedimentos}

\subsection{Fluxo das Atividades}

\subsubsection{Atividades de Desenvolvimento}

\subsubsection{Pesquisa-ação}

\subsubsection{Análise de Resultados}

\subsection{Cronograma}

\subsection{Resumo do Capítulo}

          
Do ponto de vista dos procedimentos técnicos esta é uma pesquisa 
experimental que procura estabelecer uma relação entre as causas e os efeitos de um determinado fenômino. Este fenômeno seria, os efeitos do ciclo menstrual femino na performance individual das mulheres nas tarefas cotididanas.

Esta pesquisa é de natureza aplicada por ter como objetivo final, gerar um aplicativo com sistema de recomendação baseado no perfil individual das mulheres e nas fases do ciclo menstrual femino. A forma de abordagem do problema qualitativa e quantitativa. Quanto a abordagem quantitativa, a pesquisa tem a finalidade de ser exploratória utilizando fontes bibliográficas como instrumento de conhecimento e base para afirmações. Quanto a abordagem qualitativa a finalidade da pesquisa se torna explicativa por procura identificar os fatores que contribuem ou determinam a ocorrência de fenômenose e explicar o porquê das coisas e suas causas(GIL, 2010).


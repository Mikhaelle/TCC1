\chapter[Suporte Tecnológico]{Suporte Tecnológico}

\section{Desenvolvimento da Aplicação}

\subsection{Flutter}

O Fluter é um kit de ferramenas de interface de usuário(UI) grátis e open-souce criado pela Google e lançado em 2017. Ele auxilia na criação de aplicaivos nativos para dispositivos mobile, web e desktop a partir de uma única base de código. Isso significa que é possível criar uma aplicação para diferentes sistemas operacionais(IOS e Android) utilizando um unico código \cite{flutter2017}.

Ele é composto por um SDK(Software Development Kit) e um framework. O SDK é uma coleção de ferramentas que ajudam o desenvolvedor a desenvolver a aplicação e executá-la em plataformas específicas. Essas ferramentas incluem bibliotecas, documentação, exemplos de códigos, processos, guias, compiladores, entre outras coisas. Já o framework é uma coleção de elementos da UI que são reutilizaveis e podem ser personalizados para as necessidades expecíficas da aplicação\cite{flutter2017}.

\subsubsection{Dart}

Dart é a linguagem de programação utilizada no flutter e também foi criada pela Google em 2011. É uma linguagem focada para desenvolvimento front-end e do tipo orientada a objeto\cite{flutter2017}.

\subsubsection{Firebase}

Firebase é uma plataforma desenvolvida pelo Google para a criação de aplicativos web e móveis. Era originalmente uma empresa independente fundada em 2011. Em 2014, o Google adquiriu a plataforma e agora é sua ferramenta principal para o desenvolvimento de aplicativos. O Firebase contêm funcionalidade como, análises, bancos de dados, mensagens e relatórios de erros, garantindo mais agilidade no desenvolvimento de aplicativos\cite{firebase2011}.

\section{Engenharia de Software}

\subsection{Gerenciamento do Projeto}

\subsubsection{Trello}

O Trello é uma aplicação web baseado no sistema Kanbam que auxilia no gerenciamento de tarefas para times grandes ou pessoas individuais. Originalmente criado pela Fog Creek Software em 2011 e vendida a Atlassian em 2017 \cite{trello2011}. O Trello foi utilizado para organizar as tarefas relacionadas a escrita do TCC e também para manter o registro dos artigos aqui utilizados.

\subsubsection{ZenHub}

O Zenhub é uma ferramenta semelhante ao Trello. É exclusiva para uso junto ao Github e é um poderosa ferramenta para rastreamento, planejamento e relatórios das features de projetos no GitHub. Ele é baseado nas metodologias ageis, como scrum, e é utilizado em projetos ágeis. Com ele é possível planejar roteiros, usar quadros de tarefas e gerar relatórios automatizados diretamente do repositório do GitHub \cite{zenhub2020}. Nesse projeto será utilizado para organizar as tarefas relacionadas ao desenvolvimento do aplicativo Mina.


\subsubsection{Slack}

O Slack é uma plataforma de comunicação que permite a criação de times e a organização de canais de conversas por tópicos, grupos privados ou mensagens diretas. Ele também possui integração com  Google Drive, Trello, Dropbox, Box, Heroku, IBM Bluemix, Crashlytics, GitHub, entre outros \cite{slack2013}. Nesse projeto será utilizado para comunicação entre aluno e orientador.

\subsection{Gerenciamento de Desenvolvimento}

\subsubsection{Visual Studio Code}

Visual Studio Code é um editor de texto ou código fonte feito pela Microsoft bastante utilizado no desevolvimento de software \cite{vscode2015}. Tem suporte para várias linguagens e possui ferramentas importantes que auxiliam no desenvolvimento de software, como por exemplo debugger, auto complete em códigos com sugestões inteligentes e uma infinidade de extensões que podem ser instaladas para auxiliar ainda mais no desenvolvimento. Será usado como editor de código fonte principal no trabalho. 

\subsubsection{Android Studio}

O Android Studio \cite{android2020} é o ambiente de desenvolvimento para aplicativos androids oficial da Google. Desenvolvido pela Google e pela JetBrains, o ambiente possui ferramentas como editor de texto, editor de layout, analizador apk, rápidos emuladores, debugger, entre outras ferramentas que auxiliam no desenolvimento de aplicativos. Apenas o emulador de dispositivo desse software será utilizado no trabalho por conter tanto dispositivos Android quanto IOS.

\subsubsection{Linux Mint}
O Linux Mint \cite{lm2020} é uma distribuição linux gratuita e baseada no ubuntu. O sistema operacional possui aplicações de código aberto ou código livre e é mantido pelo Linux Mint Team e a comunidade. Será usado como principal sistema operacional nesse trabalho.

\subsubsection{Figma}

O Figma \cite{figma2020} é um software para desenvolvimento de protótipos e criação de designes que não precisa de instalação porque está disponível nas nuvens através de uma página web. Possui features para design, prototipação, design de sistemas, colaboração e dowload. Os protótipos contam com interações, transições avançadas com animações inteligentes, GIFs animados, entre outros, que possibilitam uma experiência muito próxima ao mundo real. Ele permite o compartilhamento rápido do protótipo a outras pessoas e facilita a contribuição. Nesse trabalho será utilizado como ferramenta para a prototipação do aplicativo Mina. 

\subsection{Gerenciamento de Configuração}

\subsubsection{Git}

O Git \cite{git2020} é uma ferramente gratuita e de código livre para controle de versão utilizada para lidar desde pequenos a grandes projetos com rapidez e eficiência. Ele possibilita a criação de diversas branchs que se ramificam e podem ser editadas, combinadas e deletadas sem sofrer perdas de dados. O controle de versão também possibilita retornar a um momento específico do projeto e observar as mudanças entre as versões. O Git 2.17.1 é utilizado para versionamento de código fonte do aplicativo Mina e para a parte escrita do TCC.


\subsubsection{GitHub}

O GitHub \cite{github2020} é uma plataforma online de desenvolvimento de software que permite hospedar códigos e utilizar o controle de versão do Git. Possibilita a revisão de códigos, gerenciamento de projetos, integração continua, hospedagem, integrações de ferramentas, gerenciamento de equipe, documentação, hospedagem de código, entre outras funcionalidades. Os repositórios, locais aonde os projetos são guardados, podem ser fechados ou abertos e possibilitam o trabalho em equipe. Será utilizado como plataforma que hospedara o código, documentação e TCC do projeto

\section{Escrita e Condução da Pesquisa}

\subsection{LaTex}

LaTex \cite{latex2020} é um sistema para preparação de documento que inclui recursos destinados à produção de documentação técnica e científica. O LaTex é utilizado como padão para comunicação e publicação de documentos científicos e está disponível como software livre. O LaTex possibilita a composição de artigos de periódicos, relatórios técnicos, livros e apresentações de slides, faz o controle automático de seções, referências, tabelas, figuras, notas de rodapé, índices, entre outros e possíbilita a tipografia de fórmulas matemáticas complexas. Neste trabalho é utilizado localmente junto ao editor de texto padrão do sistema operacional. O template utilizado é o disponibilizado pela faculdade do gama no repositório do GitHub \footnote{Repositório do Github com o Template: https://github.com/fga-unb/template-latex-tcc}.

\section{Resumo do Capítulo}

Neste capítulo há uma breve descrição de todas as ferramentas que serão utilizadas no desenvolvimento do projeto. 

No desenvolvimento do aplicativo, será utilizado o Linux Mint como sistema operacional, o Visual Studio Code como editor de texto, e o Android Studio como o emulador de dispositivos IOS e Android e na aplicação será utilizado o Flutter como framework para criação de aplicativos nativos, o dart como linguagem de programação do Flutter e o Firebase como banco de dados e controle de acesso. 

Toda a parte de código, documentação e escrita do trabalho será disponibilizada no Github que utiliza o Git para controle de versão.

O gerenciamento do projeto do aplicativo será feito através do Zenhub, aplicativo disponível no Github, e o Trello será utilizado para o gerenciamento das tarefas do TCC. 

A escrita do TCC será frealizada utilizando o LaTex em cima de um template disponibilizado pela Faculdade do Gama em um repositório do GitHub. 

